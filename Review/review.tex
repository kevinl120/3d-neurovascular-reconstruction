%%%%%%%%%%%%%%%%%%%%%%%%%%%%%%%%%%%%%%%%%%%%%%%%%%%%%%%%%%%%%%%%%%%%%%%%%%%%%%%%
%2345678901234567890123456789012345678901234567890123456789012345678901234567890
%        1         2         3         4         5         6         7         8

% TEMPLATE for UCLA CS168

\documentclass[letterpaper, 10 pt, journal]{ieeeconf} 

\overrideIEEEmargins
% See the \addtolength command later in the file to balance the column lengths
% on the last page of the document

%\usepackage{graphics} % for pdf, bitmapped graphics files
%\usepackage{epsfig} % for postscript graphics files
%\usepackage{mathptmx} % assumes new font selection scheme installed
%\usepackage{times} % assumes new font selection scheme installed
%\usepackage{amsmath} % assumes amsmath package installed
%\usepackage{amssymb}  % assumes amsmath package installed
\usepackage{biblatex}
\addbibresource{bibfile.bib}

\title{\LARGE \bf ML Reconstruction of 3D Neurovascular Models from Biplane Views}

\author{Team \#36: Connor Chin, Kevin Li}

\begin{document}

\maketitle
\thispagestyle{empty}
\pagestyle{empty}

\section{Introduction}

Diseases in the cerebrovascular network, such as strokes, aneurysms, and Alzheimer’s are some of the leading causes of death and disabilities worldwide.
In treatment planning, it is often essential to create a 3D model of the vascular network, which can be used for segmentation, classification, or anatomical object localization.
Currently, the techniques of magnetic resonance (MR) and 3D computed tomographic (CT) angiography can produce three-dimensional volume data directly;
however, they can also bring a prohibitive set of disadvantages and inaccessibilities.
Rather, digital subtraction angiography using X-rays is more commonly used to produce high resolution images, but they may be difficult to decipher because they are only two-dimensional single or biplane projections.
In this review, we propose a deep-learning-based method for 3D reconstruction of blood vessels in cerebrovascular imagery.


\section{Related Works}

In the field of medical image processing, many methods have been developed for segmentation of the cerebrovascular network: minimum path-finding, thresholding, extraction using morphology, statistical approaches, among others and various hybrids.

Region-growing methods are computationally efficient propagation algorithms that require a starting seed and assumptions about vessel shape and size, working best with well-contrasted vessels \cite{region_growing}.
Path-finding algorithms can accurately detect vessel segments by computing a minimum intensity and connectedness criteria \cite{path_finding}.
They, however, can be computationally expensive and generally compute vessel segments rather than an entire vascular tree.
Morphological filtering is a theory of non-linear analysis that is robust to noise.
They can be used in conjunction with other criteria or classification strategies to segment vascular structures automatically \cite{morphology}.

While by no means a comprehensive review, (see \cite{segmentation_all} for further details), we observe that most segmentation methods are heuristic- or rule-based and make assumptions about the underlying structure and shape of vessels.
Additionally, these algorithms often require manual intervention at some point, whether in requiring seeds or post-processing of the graph.
Most importantly, these segmentation algorithms tend to accurately enforce local vessel constraints while ignoring global vascular network motifs.
Therefore, it is often advantageous to reconstruct the entire vascular structure rather than segment portions of it.

We note that there have been attempts to alleviate the aforementioned weaknesses, such as with the probabilistic model of Rempfler et al. \cite{reconstruction_probability}.
Recently, deep learning methods have been applied to segmentation of retinal blood vessels with reasonable success, using a standard CNN design \cite{segmentation_retinal}.
Sanches et al. \cite{segmentation_dl} proposed a cross between U-net and Inception for segmentation, titled Uception, which outperformed the original U-net.
However, to the best of our knowledge, there exist no deep learning designs dedicated to reconstructing the cerebrovascular network in entirety.

The current state-of-the-art deep learning 3D-reconstruction models include variants of MVS, 3D-R\textsuperscript{2}N\textsuperscript{2}, and 3D VAE/GAN models.

Multi-view Stereo (MVS) methods attempt to reconstruct 3D images from 2D projections at known camera angles.
Huang et. al designed DeepMVS, a deep CNN with many preprocessing techniques to outperform conventional MVS algorithms such as COLMAP \cite{deepmvs}.
The conventional MVS algorithms tends to produce results that are distored or noisy, while DeepMVS occasionally produces jagged edges.
Overall, DeepMVS shows a quantitative improvement over conventional methods.

3D-R\textsuperscript{2}N\textsuperscript{2} uses three components: a 2D CNN, a 3D convolutional LSTM (a novel architecture), and a 3D-DCNN \cite{3d-r2n2}.
Its primary advantages are that it unifies single- and multi-view reconstruction, and that it requires minimal supervision in training and testing.
Reportedly, it outperforms current state-of-the-art models in single-view reconstruction.

Finally, standard VAE/GAN methods have also been extended into three dimensions \cite{3dvaegan}.
Results show that they can be effective in single image reconstruction, though this was tested with classes of objects.
As with traditional VAE/GANs, they are able to generate multiple plausible outputs for a given input.

\section{Proposal}

From a methodical standpoint, our motivation is to implement a deep-learning model dedicated to reconstructing the full cerebrovascular network.
We hypothesize that using the current state-of-the-art 3D reconstruction neural nets will achieve more reliable results than the methods mentioned previously.
While 3D-R\textsuperscript{2}N\textsuperscript{2} achieves the highest performance single-view reconstruction, their three separate components increases training time and difficulty of hyperparameter tuning.
It is potentially possible to simplify the recurrent component, as we are primarily working with single or biplane views.
Though, the preliminary design inclination is toward the 3D VAE/GAN, due to their simplicity and performance in single-view reconstruction.



\cleardoublepage

\printbibliography
% \begin{thebibliography}{99}

% \bibitem{c1} G. O. Young, Synthetic structure of industrial plastics (Book style with paper title and editor),Ó 	in Plastics, 2nd ed. vol. 3, J. Peters, Ed.  New York: McGraw-Hill, 1964, pp. 15Ð64.

% \bibitem{c2} Pedro Sanches and Cyril Meyer and Vincent Vigon and Beno{\^{\i}}t Naegel, "Cerebrovascular Network Segmentation on {MRA} Images with Deep Learning", \textit{CoRR}, 2018.
% \bibitem{c2} https://academic.oup.com/neurosurgery/article-abstract/41/2/403/2856323
% \bibitem{c3} https://www.osti.gov/pages/servlets/purl/1245820
% \bibitem{c4} https://www.ncbi.nlm.nih.gov/pmc/articles/PMC3132942/
% \bibitem{c5} 

% \end{thebibliography}

\end{document}
